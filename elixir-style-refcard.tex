\documentclass[10pt,landscape]{article}
\usepackage{multicol}
\usepackage{calc}
\usepackage{ifthen}
\usepackage[landscape]{geometry}
\usepackage{hyperref}
\usepackage{etoolbox}
\usepackage{xcolor}
\usepackage{enumitem}
\usepackage{array}

% To make this come out properly in landscape mode, do one of the following
% 1.
%  pdflatex elixir_style_refcard.tex
%
% 2.
%  latex elixir_style_refcard.tex
%  dvips -P pdf  -t landscape elixir_style_refcard.dvi
%  ps2pdf elixir_style_refcard.ps

% The following layout is heavily based on the ``LaTex cheat Sheet'':
%
% https://github.com/wch/latexsheet

% To Do:
% \listoffigures \listoftables
% \setcounter{secnumdepth}{0}

% This sets page margins to .25 inch if using letter paper, and to .5cm
% if using A4 paper. (This probably isn't strictly necessary.)
% If using another size paper, use default 1cm margins.
\ifthenelse{\lengthtest { \paperwidth = 11in}}
	{ \geometry{top=.25in,left=.25in,right=.25in,bottom=.25in} }
	{\ifthenelse{ \lengthtest{ \paperwidth = 297mm}}
		{\geometry{top=.5cm,left=.5cm,right=.5cm,bottom=.5cm} }
		{\geometry{top=1cm,left=1cm,right=1cm,bottom=1cm} }
	}

% Turn off header and footer
\pagestyle{empty}


% Redefine section commands to use less space
\makeatletter
\renewcommand{\section}{\@startsection{section}{1}{0mm}%
                                {-1ex plus -.5ex minus -.2ex}%
                                {0.5ex plus .2ex}%x
                                {\normalfont\large\bfseries}}
\renewcommand{\subsection}{\@startsection{subsection}{2}{0mm}%
                                {-1explus -.5ex minus -.2ex}%
                                {0.5ex plus .2ex}%
                                {\normalfont\normalsize\bfseries}}
\renewcommand{\subsubsection}{\@startsection{subsubsection}{3}{0mm}%
                                {-1ex plus -.5ex minus -.2ex}%
                                {1ex plus .2ex}%
                                {\normalfont\small\bfseries}}

\setlist[itemize]{noitemsep, nosep}

% Less space in verbatim environment
\preto{\@verbatim}{\topsep=0pt \partopsep=0pt}

% Adjust fontsize for verbatim environment
\newcommand{\verbatimfont}[1]{\renewcommand{\verbatim@font}{\ttfamily#1}}
\makeatother

% Define BibTeX command
\def\BibTeX{{\rm B\kern-.05em{\sc i\kern-.025em b}\kern-.08em
    T\kern-.1667em\lower.7ex\hbox{E}\kern-.125emX}}

% Don't print section numbers
\setcounter{secnumdepth}{0}


\setlength{\parindent}{0pt}
\setlength{\parskip}{0pt plus 0.5ex}

% Colors
% \definecolor{mElixir}{HTML}{4E2A8E}


% -----------------------------------------------------------------------

\begin{document}

\raggedright
\footnotesize
\begin{multicols}{3}


% multicol parameters
% These lengths are set only within the two main columns
%\setlength{\columnseprule}{0.25pt}
\setlength{\premulticols}{1pt}
\setlength{\postmulticols}{1pt}
\setlength{\multicolsep}{1pt}
\setlength{\columnsep}{2pt}

\begin{center}
     \Large{\textbf{Elixir Style reference card}} \\
\end{center}

\section{Source Code Layout}

\begin{itemize}
  \item Use two \textbf{spaces} per indentation level.  No hard tabs.
  \item Use one expression per line, don't use semicolon \texttt{;} to separate statements and expressions.
  \item Use spaces around default arguments \verb!\\! definition.
  \item Add underscores to large numeric literals to improve their readability, e.g. \verb!num = 1_000_000!
  \item Avoid trailing whitespaces.
  \item End each file with a newline.
\end{itemize}

\subsection{Spaces in code}

Use spaces around binary operators, after commas \texttt{,}, colons \texttt{:}
and semicolons \texttt{;}. Do not put spaces around matched pairs like brackets
\texttt{[]}, braces \texttt{\{\}}, etc. Whitespace might be (mostly) irrelevant
to the Elixir compiler, but its proper use is the key to writing easily readable
code.

\begin{verbatim}
sum = 1 + 2
[first | rest] = 'three'
{a1, a2} = {2, 3}
Enum.join(["one", <<"two">>, sum])
\end{verbatim}

\subsection{No spaces in code}

No spaces after unary operators and inside range literals, the only exception is
the \texttt{not} operator.

\begin{verbatim}
angle = -45
^result = Float.parse("42.01")
2 in 1..5
not File.exists?(path)
\end{verbatim}

\subsection{bitstring segment options}

Do not put spaces around segment options definition in bitstrings.

\begin{verbatim}
# Bad
<<102 :: unsigned-big-integer, rest :: binary>>
<<102::unsigned - big - integer, rest::binary>>
# Good
<<102::unsigned-big-integer, rest::binary>>
\end{verbatim}

\subsection{Guard clauses}

Indent \texttt{when} guard clauses on the same level as the function/macro
signature in the definition they're part of. Do this only if you cannot fit the
\texttt{when} guard on the same line as the definition.

\verbatimfont{\scriptsize}
\begin{verbatim}
def format_error({exception, stacktrace})
    when is_list(stacktrace) and stacktrace != [] do
# ...
end

defmacro dngettext(domain, msgid, msgid_plural, count)
         when is_binary(msgid) and is_binary(msgid_plural) do
# ...
end
\end{verbatim}

\subsection{Multi-line expression assignment}

When assigning the result of a multi-line expression, do not preserve alignment
of its parts.

\begin{verbatim}
# Bad
{found, not_found} = Enum.map(files, &Path.expand(&1, path))
                     |> Enum.partition(&File.exists?/1)

prefix = case base do
          :binary -> "0b"
          :octal -> "0o"
          :hex -> "0x"
        end
# Good
{found, not_found} =
  Enum.map(files, &Path.expand(&1, path))
  |> Enum.partition(&File.exists?/1)

prefix = case base do
  :binary -> "0b"
  :octal -> "0o"
  :hex -> "0x"
end
\end{verbatim}

\subsection{Quotes around atoms}

When using atom literals that need to be quoted because they contain characters
that are invalid in atoms (such as \texttt{:"foo-bar"}), use double quotes
around the atom name:

\begin{tabular}{l|l}
\textit{Bad} & \textit{Good} \\
\begin{minipage}{.4\linewidth}
\begin{verbatim}
:'foo-bar'
:'atom number #{index}'
\end{verbatim}
\end{minipage}
&
\begin{minipage}{.6\linewidth}
\begin{verbatim}
:"foo-bar"
:"atom number #{index}"
\end{verbatim}
\end{minipage}
\end{tabular}

\subsection{Trailing comma}
\begin{tabular}{m{.8\linewidth}l}
When dealing with lists, maps, structs, or tuples whose elements span over
multiple lines and are on separate lines with regard to the enclosing brackets,
it's advised to use a trailing comma even for the last element.
&
\begin{minipage}{.1\linewidth}
\begin{verbatim}
[
  :foo,
  :bar,
  :baz,
]
\end{verbatim}
\end{minipage}
\end{tabular}

\subsection{Expression group alignment}
Avoid aligning expression groups:

\begin{tabular}{l|l}
\textit{Bad} & \textit{Good} \\
\begin{minipage}{.47\linewidth}
\verbatimfont{\scriptsize}
\begin{verbatim}
module = env.module
arity  = length(args)

def inspect(false), do: "false"
def inspect(true),  do: "true"
def inspect(nil),   do: "nil"
\end{verbatim}
\end{minipage}
&
\begin{minipage}{.53\linewidth}
\verbatimfont{\scriptsize}
\begin{verbatim}
module = env.module
arity = length(args)

def inspect(false), do: "false"
def inspect(true), do: "true"
def inspect(nil), do: "nil"
\end{verbatim}
\end{minipage}
\end{tabular}

The same non-alignment rule applies to \texttt{<-} and \texttt{->} clauses as
well.

\section{Syntax}

\subsection{Function parentheses}
Always use parentheses around \texttt{def} arguments, don't omit them even when
a function has no arguments.

\begin{tabular}{l|l}
\textit{Bad} & \textit{Good} \\
\begin{minipage}{.4\linewidth}
\begin{verbatim}
def main arg1, arg2 do
  #...
end

def main do
  #...
end
\end{verbatim}
\end{minipage}
&
\begin{minipage}{.6\linewidth}
\begin{verbatim}
def main(arg1, arg2) do
  #...
end

def main() do
  #...
end
\end{verbatim}
\end{minipage}
\end{tabular}

\subsection{Parentheses for local zero-arity functions}

Parentheses are a must for \textbf{local} zero-arity function calls and
definitions.

\begin{tabular}{l|l}
\textit{Bad} & \textit{Good} \\
\begin{minipage}{.4\linewidth}
\begin{verbatim}
pid = self
def new, do: %MapSet{}
\end{verbatim}
\end{minipage}
&
\begin{minipage}{.6\linewidth}
\begin{verbatim}
pid = self()
def new(), do: %MapSet{}
config = IEx.Config.new
\end{verbatim}
\end{minipage}
\end{tabular}

The same applies to \textbf{local} one-arity function calls in pipelines.

\begin{verbatim}
input |> String.strip() |> decode()
\end{verbatim}

\subsection{Anonymous function parentheses}

Never wrap the arguments of anonymous functions in parentheses.

\begin{tabular}{l|l}
\textit{Bad} & \textit{Good} \\
\begin{minipage}{.45\linewidth}
\verbatimfont{\tiny}
\begin{verbatim}
Agent.get(pid, fn(state) -> state end)
Enum.reduce(numbers, fn(number, acc) ->
  acc + number
end)
\end{verbatim}
\end{minipage}
&
\begin{minipage}{.55\linewidth}
\verbatimfont{\tiny}
\begin{verbatim}
Agent.get(pid, fn state -> state end)
Enum.reduce(numbers, fn number, acc ->
  acc + number
end)
\end{verbatim}
\end{minipage}
\end{tabular}

\subsection{Pipeline operator}

Favor the pipeline operator \texttt{|>} to chain function calls together.

\begin{tabular}{l|l}
\textit{Bad} & \textit{Good} \\
\begin{minipage}{.4\linewidth}
\verbatimfont{\tiny}
\begin{verbatim}
String.downcase(String.strip(input))
\end{verbatim}
\end{minipage}
&
\begin{minipage}{.6\linewidth}
\verbatimfont{\tiny}
\begin{verbatim}
input |> String.strip |> String.downcase
String.strip(input) |> String.downcase
\end{verbatim}
\end{minipage}
\end{tabular}

Use a single level of indentation for multi-line pipelines.

\begin{verbatim}
String.strip(input)
|> String.downcase
|> String.slice(1, 3)
\end{verbatim}

\subsection{Needless pipeline operator}

Avoid needless pipelines like the plague.

\begin{tabular}{l|l}
\textit{Bad} & \textit{Good} \\
\begin{minipage}{.5\linewidth}
\verbatimfont{\scriptsize}
\begin{verbatim}
result = input |> String.strip
\end{verbatim}
\end{minipage}
&
\begin{minipage}{.5\linewidth}
\verbatimfont{\scriptsize}
\begin{verbatim}
result = String.strip(input)
\end{verbatim}
\end{minipage}
\end{tabular}

\subsection{Binary operators at the end of line}

When writing a multi-line expression, keep binary operators at the end of each
line. The only exception is the \texttt{|>} operator (which goes at the
beginning of the line).

\begin{tabular}{l|l}
\textit{Bad} & \textit{Good} \\
\begin{minipage}{.4\linewidth}
\begin{verbatim}
"No matching message.\n"
<> "Process mailbox:\n"
<> mailbox
\end{verbatim}
\end{minipage}
&
\begin{minipage}{.6\linewidth}
\begin{verbatim}
"No matching message.\n" <>
"Process mailbox:\n" <>
mailbox
\end{verbatim}
\end{minipage}
\end{tabular}

\subsection{\texttt{with} indentation}

Use the indentation shown below for the \texttt{with} special form:

\begin{verbatim}
with {year, ""} <- Integer.parse(year),
     {month, ""} <- Integer.parse(month),
     {day, ""} <- Integer.parse(day) do
  new(year, month, day)
else
  _ ->
    {:error, :invalid_format}
end
\end{verbatim}

Always use the indentation above if there's an \texttt{else} option. If there
isn't, the following indentation works as well:

\begin{verbatim}
with {:ok, date} <- Calendar.ISO.date(year, month, day),
     {:ok, time} <- Time.new(hour, minute, second, microsecond),
     do: new(date, time)
\end{verbatim}

\subsection{\texttt{for} indentation}

\begin{verbatim}
for {alias, _module} <- aliases_from_env(server),
    [name] = Module.split(alias),
    starts_with?(name, hint),
    into: [] do
  %{kind: :module, type: :alias, name: name}
end
\end{verbatim}

If the body of the \texttt{do} block is short, the following indentation works
as well:

\begin{verbatim}
for partition <- 0..(partitions - 1),
    pair <- safe_lookup(registry, partition, key),
    into: [],
    do: pair
\end{verbatim}

\subsection{Never use \texttt{unless} with \texttt{else}}

Rewrite these with the positive case first.

\begin{tabular}{l|l}
\textit{Bad} & \textit{Good} \\
\begin{minipage}{.45\linewidth}
\begin{verbatim}
unless Enum.empty?(coll) do
  :ok
else
  :error
end
\end{verbatim}
\end{minipage}
&
\begin{minipage}{.55\linewidth}
\begin{verbatim}
if Enum.empty?(coll) do
  :error
else
  :ok
end
\end{verbatim}
\end{minipage}
\end{tabular}

\subsection{No \texttt{nil}-\texttt{else}}

Omit \texttt{else} option in \texttt{if} and \texttt{unless} clauses if it
returns \texttt{nil}.

\begin{verbatim}
# Bad
if byte_size(data) > 0, do: data, else: nil

# Good
if byte_size(data) > 0, do: data
\end{verbatim}

\subsection{\texttt{true} in \texttt{cond}}

If you have an always-matching clause in the \texttt{cond} special form, use
\texttt{true} as its condition.

\begin{tabular}{l|l}
\textit{Bad} & \textit{Good} \\
\begin{minipage}{.4\linewidth}
\begin{verbatim}
cond do
  char in ?0..?9 ->
    char - ?0
  char in ?A..?Z ->
    char - ?A + 10
  :other ->
    char - ?a + 10
end
\end{verbatim}
\end{minipage}
&
\begin{minipage}{.6\linewidth}
\begin{verbatim}
cond do
  char in ?0..?9 ->
    char - ?0
  char in ?A..?Z ->
    char - ?A + 10
  true ->
    char - ?a + 10
end
\end{verbatim}
\end{minipage}
\end{tabular}

\subsection{Boolean Operators}

Never use \texttt{||}, \verb!&&! and \texttt{!} for strictly boolean checks.
Use these operators only if any of the arguments are non-boolean.

\begin{tabular}{l|l}
\textit{Bad} & \textit{Good} \\
\begin{minipage}{.4\linewidth}
\verbatimfont{\tiny}
\begin{verbatim}
is_atom(name) && name != nil
is_binary(task) || is_atom(task)
\end{verbatim}
\end{minipage}
&
\begin{minipage}{.6\linewidth}
\verbatimfont{\tiny}
\begin{verbatim}
is_atom(name) and name != nil
is_binary(task) or is_atom(task)
line && line != 0
file || "sample.exs"
\end{verbatim}
\end{minipage}
\end{tabular}

\subsection{Patterns matching binaries}

Favor the binary concatenation operator \texttt{<>} over bitstring syntax for
patterns matching binaries.

\begin{tabular}{l|l}
\textit{Bad} & \textit{Good} \\
\begin{minipage}{.4\linewidth}
\verbatimfont{\tiny}
\begin{verbatim}
<<"http://", _rest::bytes>> = input
<<first::utf8, rest::bytes>> = input
\end{verbatim}
\end{minipage}
&
\begin{minipage}{.6\linewidth}
\verbatimfont{\tiny}
\begin{verbatim}
"http://" <> _rest = input
<<first::utf8>> <> rest = input
\end{verbatim}
\end{minipage}
\end{tabular}

\subsection{Use uppercase in definition of hex literals}

\begin{tabular}{l|l}
\textit{Bad} & \textit{Good} \\
\begin{minipage}{.4\linewidth}
\begin{verbatim}
<<0xef, 0xbb, 0xbf>>
\end{verbatim}
\end{minipage}
&
\begin{minipage}{.6\linewidth}
\begin{verbatim}
<<0xEF, 0xBB, 0xBF>>
\end{verbatim}
\end{minipage}
\end{tabular}

\section{Naming}

\begin{itemize}
  \item Use \verb!snake_case! for naming directories and files, e.g.
\verb!lib/my_app/task_server.ex!.
  \item Avoid using one-letter variable names.
  \item Use \verb!snake_case! for atoms, functions, variables and module attributes.
\end{itemize}

\begin{tabular}{l|l}
\textit{Bad} & \textit{Good} \\
\begin{minipage}{.4\linewidth}
\begin{verbatim}
:"no match"
:Error
:badReturn

fileName = "sample.txt"

@_VERSION "0.0.1"

def readFile(path) do
  #...
end
\end{verbatim}
\end{minipage}
&
\begin{minipage}{.6\linewidth}
\begin{verbatim}
:no_match
:error
:bad_return

file_name = "sample.txt"

@version "0.0.1"

def read_file(path) do
  #...
end
\end{verbatim}
\end{minipage}
\end{tabular}

\subsection{Use CamelCase for module names}

\begin{tabular}{l|l}
\textit{Bad} & \textit{Good} \\
\begin{minipage}{.4\linewidth}
\begin{verbatim}
defmodule :appStack do
#...
end

defmodule App_Stack do
#...
end

defmodule Appstack do
#...
end
\end{verbatim}
\end{minipage}
&
\begin{minipage}{.6\linewidth}
\begin{verbatim}
defmodule AppStack do
#...
end
\end{verbatim}
\end{minipage}
\end{tabular}

\subsection{Predicate function names}

\begin{tabular}{ll}
\begin{minipage}{.5\linewidth}
The names of predicate functions (a function that return a boolean value) should
have a trailing question mark \texttt{?} rather than a leading \verb!has_! or
similar.

\begin{verbatim}
def leap?(year) do
#...
end
\end{verbatim}
\end{minipage}
&
\begin{minipage}{.4\linewidth}
Always use a leading \verb!is_! when naming guard-safe predicate macros.

\verbatimfont{\tiny}
\begin{verbatim}
defmacro is_date(month, day) do
#...
end
\end{verbatim}
\end{minipage}
\end{tabular}

\section{Comments}

\begin{itemize}
  \item Write self-documenting code and ignore the rest of this section.
        \textbf{Seriously!}
  \item Use one space between the leading \verb!#! character of the comment
        and the text of the comment.
  \item Avoid superfluous comments. e.g. \verb!String.first(input) # Get first grapheme!
\end{itemize}

\section{Modules}

\subsection{Module Layout}

Use a consistent structure when calling \texttt{use}/\texttt{import}/\texttt{alias}/\texttt{require}:
call them in this order and group multiple calls to each of them.

\begin{verbatim}
use GenServer

import Bitwise
import Kernel, except: [length: 1]

alias Mix.Utils
alias MapSet, as: Set

require Logger
\end{verbatim}

\subsection{Use \texttt{\_\_MODULE\_\_} to reference current module}

\begin{verbatim}
# Bad
:ets.new(Kernel.LexicalTracker, [:named_table])
GenServer.start_link(Module.LocalsTracker, nil, [])

# Good
:ets.new(__MODULE__, [:named_table])
GenServer.start_link(__MODULE__, nil, [])
\end{verbatim}

\subsection{Regular expressions are the last resort}

Pattern matching and \texttt{String} module are things to start with.

\begin{verbatim}
# Bad
Regex.run(~r/#(\d{2})(\d{2})(\d{2})/, color)
Regex.match?(~r/(email|password)/, input)

# Good
<<?#, p1::2-bytes, p2::2-bytes, p3::2-bytes>> = color
String.contains?(input, ["email", "password"])
\end{verbatim}

\subsection{Use non-capturing RegEx}

\ldots when you don't use the captured result.

\begin{verbatim}
~r/(?:post|zip )code: (\d+)/
\end{verbatim}

\subsection{Caret and dollar RegEx}

Be careful with \verb!^! and \verb!$! as they match start and end of the
\textbf{line} respectively. If you want to match the \textbf{whole} string use:
\verb!\A! and \verb!\z! (not to be confused
with \verb!\Z! which is the equivalent of
\verb!\n?\z!).

\section{Structs}

\subsection{\texttt{defstruct} with default fields}

When calling \texttt{defstruct/1}, don't explicitly specify \texttt{nil} for
fields that default to \texttt{nil}.

\begin{verbatim}
# Bad
defstruct first_name: nil, last_name: nil, admin?: false

# Good
defstruct [:first_name, :last_name, admin?: false]
\end{verbatim}

\section{Exceptions}

\subsection{Make exception names end with a trailing \texttt{Error}}

\begin{tabular}{l|l}
\textit{Bad} & \textit{Good} \\
\begin{minipage}{.4\linewidth}
\begin{verbatim}
BadResponse
ResponseException
\end{verbatim}
\end{minipage}
&
\begin{minipage}{.6\linewidth}
\begin{verbatim}
ResponseError
\end{verbatim}
\end{minipage}
\end{tabular}

\subsection{Use non-capitalized error messages}

\ldots when raising exceptions, with no trailing punctuation.

\begin{verbatim}
# Bad
raise ArgumentError, "Malformed payload."

# Good
raise ArgumentError, "malformed payload"
\end{verbatim}

There is one exception to the rule - always capitalize Mix error messages.

\begin{verbatim}
Mix.raise "Could not find dependency"
\end{verbatim}

\section{Typespecs}

\subsection{Never use parens on zero-arity types}

\begin{verbatim}
# Bad
@spec start_link(module(), term(), Keyword.t()) :: on_start()

# Good
@spec start_link(module, term, Keyword.t) :: on_start
\end{verbatim}

\section{ExUnit}

\subsection{ExUnit assertion side}

When asserting (or refuting) something with comparison operators (such as
\texttt{==}, \texttt{<}, \texttt{>=}, and similar), put the  expression being
tested on the left-hand side of the operator and the value you're testing
against on the right-hand side.

\begin{verbatim}
# Bad
assert "héllo" == Atom.to_string(:"héllo")

# Good
assert Atom.to_string(:"héllo") == "héllo"
\end{verbatim}

When using the match operator \texttt{=}, put the pattern on the left-hand side
(as it won't work otherwise).

\begin{verbatim}
assert {:error, _reason} = File.stat("./non_existent_file")
\end{verbatim}

\hfill
\hrule
\hfill
\scriptsize

This reference card is an adaptation of the \href{https://github.com/lexmag/elixir-style-guide}{``Elixir Style Guide''},
it was created by Milton Mazzarri and is licensed under the
\href{https://creativecommons.org/licenses/by/4.0/}{CC BY 4.0 license}. \href{http://github.com/milmazz/elixir-style-refcard/}{http://github.com/milmazz/elixir-style-refcard/}

You can find the original ``Elixir Style Guide'' by Aleksei Magusev here: \href{https://github.com/lexmag/elixir-style-guide}{https://github.com/lexmag/elixir-style-guide}

\end{multicols}
\end{document}
